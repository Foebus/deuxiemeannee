\documentclass[a4paper,10pt]{article}
\usepackage[utf8]{inputenc}
\usepackage[french]{babel}
\usepackage{amsfonts,bm,amsmath,amssymb,graphicx,verbatim}
\usepackage{cancel}

\newcommand{\dd}{\mathrm{d}}
\newcommand{\hf}{\hat{f}}
\newcommand{\real}{\mathbb{R}}
\newcommand{\integer}{\mathbb{Z}}
\newcommand{\definition}{\textbf{Definition }}
\newcommand{\exemples}{\textbf{Exemples }}
\newcommand{\remarque}{\textbf{Remarque }}
\newcommand{\proprietes}{\textbf{Propriétés }}
\newcommand{\propriete}{\textbf{Propriété }}

\title{Travaux pratiques: Transformées de Fourier}
% \author{Orestis Malaspinas}
\date{A rendre pour le 24.04.2017}

\begin{document}
\maketitle

\section*{Travaux pratiques: Transformées de Fourier}

Le but de ce travail est d'essayer de comprendre comment utiliser les transformées de Fourier dans différentes applications,
en particulier le filtrage. Pour ce faire, nous allons considérer une fonction analytique ``simple'' (dont
la transformée de Fourier est facile à calculer) et la manipuler avec les fonctions préimplantées dans Matlab/Octave de
transformées de Fourier: \texttt{fft} et \texttt{ifft} (il s'agit donc également d'essayer de comprendre comment les manipuler). 
Ces fonctions représentent respectivement les transformées de Fourier et transformées de Fourier inverses rapides.

Dans un premier temps, nous allons considérer la fonction
\begin{equation}
 f(t)=\cos(2\pi t)+0.9\cos(2\pi 10 t).
\end{equation}
Calculez analytiquement les coefficients de la série de Fourier de cette fonction
et la dessinez pour $t=0..10$ avec $\delta t=0.025$ le pas entre deux points.

Une fois que cette étape est effectuée, utilisez la fonction \texttt{fft}, pour 
calculer la transformée de Fourier, $\hf$, de $f$. Pour ce faire,
il faut ``\'echantillonner $f$ (choisir le pas de temps qu'on veut pour représenter 
$f(t)$ numériquement). Un bon choix est de prendre $\delta t=0.025$.
Représentez le module de transformée de Fourier sur un graphique en fonction de la fréquence, $\nu$\footnote{Indication: l'amplitude de la transformée de Fourier doit être normalisée par le 
nombre échantillons de la fonction. De plus vous allez constater que le spectre se trouve représenté deux fois dans le vecteur donné par la fonction \texttt{fft}, à vous de tout remettre à l'échelle comme il faut}.
Qu'observez-vous? Le résultat est-il cohérent avec le résultat analytique? 
Reconstruire $f(t)$ à partir de $\hf(\nu)$ avec la fonction \texttt{ifft}. Superposer la fonction obtenue avec avec la fonction originale, que note-t-on?
Refaites ces étapes en utilisant $\delta t=0.05,0.1$ que notez-vous? Comment expliquer le phénomène?

En principe, vous avez dû trouver un spectre avec deux pics. Ôtez le premier 
pic, puis le second et avec \texttt{ifft} 
calculez les transformées de Fourier inverses et représentez les superposées à la fonction originale\footnote{Il faut ``filtrer'' dans le monde des fréquences et effectuer la tranformées de Fourier inverse pour avoir 
le signal dans le temps. Attention le spectre est présent à double.}. Discutez les résultats.

Chargez le fichier \texttt{mydata.txt} qui contient deux colonnes. Le temps $t$, et une fonction $h(t)$. Représentez la fonction sur un graphique.
Calculez la transformée de Fourier de cette fonction avec \texttt{fft} et faites un graphique de $\hat{h}(\nu)$. Filtrez toutes les fréquences $\nu>10$ de $\hat{h}(\nu)$ dans 
l'espace spectral. Reconstruisez la la fonction dans l'espace temporel depuis la fonction filtrée. Superposez
le résultat avec la fonction originale. Que constatez-vous? Comparez le résultat avec la fonction $f(t)$.


\section*{Remarques}

Le travail peut-être effectué en groupe de deux. 
Déposez le rapport et le code sur le site du cours s'il vous plaît, cela simplifie mon administration (et évite les problèmes avec les étudiants qui 
ne mettent pas de nom sur le rapport...).

La note sera une combinaison entre le code rendu et le rapport (moitié/moitié). 



\end{document}
